\section{Uvod}

Statistika je veda, ki se ukvarja z zbiranjem, analiziranjem, interpretacijo, predstavitvijo in organizacijo podatkov. Izhaja iz \textit{statisticum} (državni), saj je prvotno označevala analizo podatkov o državi. Njene uporabe presegajo matematiko in je temeljno orodje za raziskovanje na vseh področjih znanosti. Pomembno je, da jo uporabljamo odgovorno in etično ter kritično ocenjujemo kontekst in vir statističnih informacij, da se izognemo napačnim interpretacijam in zavajanju (zgled zavajanja).


\subsection*{Podatki in informacije}

\textbf{Podatki} so niz vrednosti kvalitativnih spremenljivk.\\
\textbf{Informacije} so obdelani in interpretirani podatki.\\
Primer: temperatura po vsem svetu za zadnjih 100 let so podatki; analiza, ki ugotavlja, da globalna temperatura narašča, je informacija


\subsection*{Enote opazovanja in Spremenljivke}

Enote opazovanja, za katere se zbirajo podatki, npr. oseba, gospodinjstvo, podjetje, izdelek, stavba, dogodek, država …\\
Spremenljivka je katero koli značilno število ali količina, ki jo je mogoče izmeriti ali prešteti, na primer:
\begin{itemize}
\item Spol, starost, izobrazba, poklic, dohodek, narodnost … (za osebo)
\item Število članov, vrsta, lastništvo stanovanja, dohodek, internetna povezava … (za gospodinjstvo)
\item Število zaposlenih, lokacija, vrsta, sektor, prihodki … (za podjetje)
\item Dimenzije, teža, starost, barva, temperatura, cena … (za izdelek)
\item Dimenzije, lokacija, starost, lastništvo, materiali, cena … (za stavbo)
\item Površina, prebivalstvo, število podjetij, politična ureditev … (za državo)
\end{itemize}


\subsection*{Tipi spremenljivk glede na vrednost in glede na mersko lestvico}

\subsubsection*{Glede na vrednost}

\textbf{Kategorične} (atributne) spremenljivke, npr. spol, izobrazba, barva, sektor, vrsta, regija\\
\textbf{Številske} spremenljivke:
\begin{itemize}
\item Zvezne (lahko imajo poljubne vrednosti), npr. točna starost, dohodek, cena, dimenzije, dolžina, širina, trajanje …
\item Diskretne (imajo le celoštevilske vrednosti), npr. letnica rojstva, velikost gospodinjstva, velikost podjetja, število udeležencev …
\end{itemize}

\subsubsection*{Glede na mersko lestvico}
\begin{itemize}
\item \textbf{Nominalne} spremenljivke: vrednosti se lahko razlikujejo le med seboj, razvrščanje ni možno, npr. spol, poklic, sektor, narodnost, regija …
\item \textbf{Ordinalne} spremenljivke: vrednosti so lahko razvrščene od najmanjše do največje, vendar razdalje med vrednostmi niso znane, npr. izobrazba, šolska ocena, stopnja strinjanja, stopnja zadovoljstva, tesnoba …
\item \textbf{Intervalne} spremenljivke: razlika med dvema vrednostma je smiselna, vendar ni dejanske ničelne vrednosti, je samo arbitrarna, npr. temperatura na lestvici Celzija, pH, koledarsko leto …
\item \textbf{Razmernostne} spremenljivke: imajo edinstveno in nearbitrarno ničelno vrednost, zato lahko izračunamo tudi razmerja, npr. temperatura po Kelvinovi lestvici, starost, dolžina, širina, višina, teža, velikost razreda, število udeležencev dogodka, dohodek …
\end{itemize}

\begin{table}[h!]
\centering
\begin{tabular}{|>{\raggedright}m{5cm}|c|c|c|c|}
        \hline
        \textbf{Kaj lahko izračunamo} & \textbf{Nominalna} & \textbf{Ordinalna} & \textbf{Intervalna} & \textbf{Razmernostna} \\
        \hline
        Frekvenčna porazdelitev & \checkmark & \checkmark & \checkmark & \checkmark \\
        \hline
        Modus & \checkmark & \checkmark & \checkmark & \checkmark \\
        \hline
        Vrsti red vrednosti & & \checkmark & \checkmark & \checkmark \\
        \hline
        Mediana & & \checkmark & \checkmark & \checkmark \\
        \hline
        Povprečje & & & \checkmark & \checkmark \\
        \hline
        Razlika med vrednostmi & & & \checkmark & \checkmark \\
        \hline
        Seštevanje in odštevanje & & & \checkmark & \checkmark \\
        \hline
        Množenje in deljenje & & & & \checkmark \\
        \hline
\end{tabular}
\end{table}

\textbf{Python Example:}

\begin{lstlisting}[language=Python]
        # Import necessary libraries
        import pandas as pd
        
        # Step 1: Read the CSV file
        # Assume the CSV file is named 'data.csv' and located in the same directory as the script
        df = pd.read_csv('data.csv')
        
        # Step 2: Print the first few rows of the dataframe
        print("First few rows of the dataframe:")
        print(df.head())
        
        # Step 3: Check the data types of each column
        print("\nData types of each column:")
        print(df.dtypes)
        
        # Step 4: Convert a specific column to float (if necessary)
        # Let's assume we have a column named 'income' which we want to convert to float
        df['income'] = df['income'].astype(float)
        
        # Verify the conversion
        print("\nData types after conversion:")
        print(df.dtypes)
        
        # Step 5: Calculate the range of data in a numeric column
        # Let's calculate the range for the 'income' column
        income_range = df['income'].max() - df['income'].min()
        print("\nRange of the 'income' column:")
        print(income_range)
        
        # Step 6: Label encoding for an ordinal categorical variable
        # Let's assume we have an ordinal variable named 'education_level'
        education_levels = {'High School': 1, 'Bachelor': 2, 'Master': 3, 'PhD': 4}
        df['education_level'] = df['education_level'].map(education_levels)
        
        # Verify the encoding
        print("\nData after label encoding 'education_level':")
        print(df.head())
        
        # Step 7: One-hot encoding for a nominal categorical variable
        # Let's assume we have a nominal variable named 'region'
        df = pd.get_dummies(df, columns=['region'], prefix='region')
        
        # Verify the one-hot encoding
        print("\nData after one-hot encoding 'region':")
        print(df.head())
        
        # Additional Step: Descriptive statistics summary
        print("\nDescriptive statistics of the dataframe:")
        print(df.describe())
        
        # Save the modified dataframe to a new CSV file
        df.to_csv('modified_data.csv', index=False)
        
        print("\nModified dataframe saved to 'modified_data.csv'.")        
\end{lstlisting}

\section*{Populacija in vzorec}

\textbf{Populacija} se nanaša na skupni niz opazovanj; pomembno jo je prostorsko in časovno opredeliti, npr.
\begin{itemize}
\item študenti Univerze na Primorskem v študijskem letu 2023/2024
\item javni vrtci v Obalno-Kraški regiji na 1. 9. 2023
\item gledališke predstave v Sloveniji v tednu od 19. do 25. 2. 2024
\item knjige izdane v EU v januarju 2024
\end{itemize}
\textbf{Vzorec} se nanaša na niz podatkov, izbranih iz statistične populacije po določenem postopku, npr.
\begin{itemize}
\item sistematični vzorec 400 študentov UP
\item naključni vzorec 200 knjig
\end{itemize}

\section*{Vrste statistične analize}

Glede na namen:
\begin{itemize}
\item Opisna (deduktivna) statistika: analiza in opis zbranih podatkov brez težnje po posploševanju teh podatkov izven njihovega obsega
\item Inferenčna (induktivna) statistika: sklepanje iz vzorca na populacijo
\end{itemize}
Glede na število sočasno analiziranih spremenljivk:
\begin{itemize}
\item Univariatna statistika: analiza ene spremenljivke
\item Bivariatna statistika: analiza dveh spremenljivk, npr. hi-kvadrat, mere povezanosti, t-test, ANOVA, regresija, …
\item Multivariatna statistika: analiza več spremenljivk, npr. multipla regresija, analiza glavnih komponent, faktorska analiza, diskriminantna analiza …
\end{itemize}

\section*{Koraki statistične analize}

\begin{enumerate}[i]
\item Določitev vsebine in namena statistične študije, opredelitev objekta (enota in populacija) in vsebino opazovanja (spremenljivke)
\item Statistično opazovanje (celotne populacije ali vzorca)
\item Enostavna obdelava (urejanje, soritranje podatkov in izračun osnovnih karakteristik)
\item Analitična obravnava
\end{enumerate}

\section*{Vaje}

\begin{Vaje}{1}
        Za naslednje spremenljivke definirajte nekaj možnih vrednosti in navedite, ali so zvezne ali diskretne, ter kakšna je raven merjenja:
        \begin{itemize}
            \item Število dnevnih poslov na ljubljanski borzi
            \item Temperatura v Kopru v stopinjah Celzija
            \item Življenjska doba osebnega računalnika
            \item Število dni letnega dopusta za zaposlene
            \item Dnevno prehojena razdalja:
            \begin{itemize}
                \item a. kilometrih
                \item b. korakih
            \end{itemize}
            \item Leto neto dohodek učitelja
            \item Teža solate v gramih
            \item Število polic v knjižni omari
            \item Strinjanje s trditvijo na lestvici od 1 (Sploh se ne strinjam) do 5 (Povsem se strinjam)
        \end{itemize}
\end{Vaje}

\begin{Vaje}{2}
        Za naslednje enote navedite nekaj primerov spremenljivk in določite njihovo mersko lestvico:
        \begin{itemize}
            \item Učenec
            \item Učitelj
            \item Razred
            \item Šola
            \item Učbenik
        \end{itemize}
\end{Vaje}

\begin{Vaje}{3}
        Predstavljajte si, da proučujete pojav besede “trajnostni razvoj” v slovenskih srednješolskih učbenikih izdanih v obdobju od 2010 do 2020. Med njimi naključno izberete 250 učbenikov, v katerih preštejete, kolikokrat se pojavi beseda trajnostni razvoj.
        \begin{itemize}
            \item Kaj je enota analize in kaj je spremenljivka?
            \item Kaj je merska lestvica spremenljivke?
            \item Kakšen je vzorec in kako velik je?
            \item Kakšna je populacija?
        \end{itemize}
\end{Vaje}

\begin{Vaje}{4}
        Recimo, da je imel Pokrajinski muzej v Kopru v letu 2023 natanko 20.000 obiskovalcev. Predstavljate si, da je vsak deseti obiskovalec muzeja prejel in izpolnil kratek vprašalnik:
        \begin{itemize}
            \item Kako velika je populacija?
            \item Kako velik je vzorec?
            \item Napiši vsaj štiri vprašanja, na podlagi katerih bi lahko definirali eno nominalno, eno ordinalno, eno intervalno in eno razmernostno spremenljivko. 
        \end{itemize}
\end{Vaje}

\begin{Vaje}{5}
        Razišči bazo podatkov raziskave PISA na spletni strani \href{https://www.oecd.org/pisa/data/2022database/}{OECD}
\end{Vaje}