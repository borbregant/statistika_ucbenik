\section{Opisna statistika in vizualizacija}

\subsection*{Absolutna, relativna in grupirana frekvenčna porazdelitev}

\subsection*{Grafična predsstavitev frelvenčnih porazdelitev za nominalne in ordinalne spremenljivke}

Stolpčni grafikon $>$ tortni grafikon

\subsection*{Grafična predsstavitev frelvenčnih porazdelitev za intervalne in razmernostne spremenljivke}

Histogram

Poligon

Ogiva (kumulativne frekvence)

\subsection*{Normalna porazdelitev}

\[f_X(x)=\frac{1}{\sigma\sqrt{2\pi}}e^{-\frac{1}{2}\frac{(x-\mu)^2}{\sigma^2}}\]

Slika

Asimetričnost (skewness)

V levo $(k<0)$, v desno $(k>0)$, če k med -1 in 1 še vedno normalna. Slika

Sploščenost (kurtosis)

Lepto $(k<0)$, mezo $(k=0)$, plati $(k>0)$, slika, če med -1,1 normalna

\subsection*{Rangiranje}

Primer

\subsection*{Kvantili}

Primer in imena

\subsection*{Mere centralne tendence}
\begin{itemize}
    \item \textbf{Modus} - vrednost, ki se najpogosteje pojavi v nizu vrednosti podatkov
    \item \textbf{Mediana} - vrednost, ki ločuje zgornjo polovico obsega razpona vrednosti od spodnje polovice
    \item \textbf{Aritmetična sredina} - povprečje niza vrednosti
    \item Druge mere (geometrijska, harmonična sredina, ...)
\end{itemize}
Primerjava med modusom, mediano in aritmetično za unimodalne asimetrične - slika

\subsection*{Mere variabilnosti (disperzije)}
V kolikšni meri se vrednosti razlikujejo med seboj ter razlikujejo in odstopajo od povprečja. Delimo jih na:
\begin{itemize}
    \item Absolutne mere (razpon, interkvartilni rang, absolutna deviacija aritmetične sredine/mediane, varianca in standardni odklon) 
    \item Relativne mere (absolutne mere deljene s pripadajočo mero centralne tendence) se izračunajo samo za razmernostne spremenljivke; uporabljamo jih, ko želimo primerjati:
    \begin{itemize}
        \item Dve porazdelitvi z zelo različno vrednostjo za nek mero centralne tendence;
        \item Dve spremenljivki z različnima merskima enotama.
    \end{itemize}
\end{itemize}


\begin{multicols}{2}
\noindent
\textbf{Theory:}

Data visualization is the graphic representation of data. It involves producing images that communicate relationships among the represented data to viewers of the images. Visualization makes it easier to detect patterns, trends, and outliers in groups of data.

\columnbreak

\textbf{Python Example:}

\begin{verbatim}
import seaborn as sns
import pandas as pd

# Load an example dataset
data = sns.load_dataset("iris")

# Create a pairplot
sns.pairplot(data, hue="species")
plt.show()
\end{verbatim}
\end{multicols}