\section{Bivariantna statistika}

\subsection*{T-test}

\subsubsection*{T-test za (parne) odvisne vzorce}

Primerjava povprečij dveh pogojev, v katerih so sodelovale iste enote.

Primer: 20 študentov je dobilo test v izpolnjevanje pred študijem določenega predmeta in nato ponovno po zaključku tega predmeta.

Testna statistika $t=\frac{\bar{d}}{SE(\bar{d})}$.

Izračun:

\begin{enumerate}
    \item Postavimo ničelno in alternativno hipotezo:
        \begin{itemize}
            \item $H_0$: Ni razlik v znanju študentov pred in po študiju tega modula.
            \item $H_1$: Obstajajo razlike v znanju študentov pred in po študiju tega modula.
        \end{itemize}
    \item Izračunamo razlike med pari opazovanj: $d_i = y_i - x_i$ (za vajo lahko to naredimo v SPSS).
    \item Izračunamo povprečje razlik: $\bar{d}$.
    \item Standardni odklon razlik: $s_d$.
    \item Standardna napaka povprečne razlike: $SE(\bar{d}) = \frac{s_d}{\sqrt{n}}$.
    \item T-statistika: $t = \frac{\bar{d}}{SE(\bar{d})}$ (empirična vrednost); $df = n-1$.
    \item V tabeli poiščemo kritično vrednost pri $\alpha = 5\%$.
    \item Empirična vrednost $t$ pade v kritično območje, ki ga določa teoretična vrednost $t$ pri dani stopnji zaupanja, zato lahko zavrnemo ničelno hipotezo in sprejmemo alternativno..... Napiši kaj če ne pade v kritično pade!!!!
    \item Interval zaupanja za resnično vrednost razlike povprečij je: 
        \[\bar{d} \pm (t \cdot SE(\bar{d}))\]
\end{enumerate}


\subsubsection*{T-test za neodvisne vzorce}

Primerjava preizkusa domneve o srednjih vrednosti dveh skupin enot.

Primer: Primerjava kalorične vrednosti dveh vrst štrudlja.

Testna statistika $t = \frac{\bar{x_1}-\bar{x_2}}{SE(\bar{x_1}-\bar{x_2})}$.

Opomba: Test predpostavi enakost varianc neodvisnih vzorcev. Če to ni zagotovoljeno, uporabimo Welchov test $t = \frac{\bar{x_1}-\bar{x_2}}{\sqrt{\frac{s_1^2}{n_1}+\frac{s_2^2}{n_2}}}$.

Izračun:
\begin{enumerate}
    \item Postavimo ničelno in alternativno hipotezo:
        \begin{itemize}
            \item $H_0$: Ni razlik v kalorični vsebnosti med dvema vrstama hotdoga.
            \item $H_1$: So razlike v kalorični vsebnosti med dvema vrstama hotdoga.
        \end{itemize}
    \item Razlika med povprečnima vrednostima: $\bar{x}_1 - \bar{x}_2$.
    \item Skupni standardni odklon (pod predpostavko enakih varianc):
        \[s_p = \sqrt{\frac{(n_1 - 1) \cdot s_1^2 + (n_2 - 1) \cdot s_2^2}{n_1 + n_2 - 2}}\]
    \item Standardna napaka: 
        \[SE(\bar{x}_1 - \bar{x}_2) = s_p \cdot \sqrt{\frac{1}{n_1} + \frac{1}{n_2}}\]
    \item Empirična t-vrednost: 
        \[t = \frac{\bar{x}_1 - \bar{x}_2}{SE(\bar{x}_1 - \bar{x}_2)}\]
    \item Prostostne stopnje: $df = n_1 + n_2 - 2$.
    \item V tabeli poiščemo kritično vrednost pri $\alpha = 5\%$.
    \item Interval zaupanja za resnično vrednost razlike povprečij: 
        \[\bar{x}_1 - \bar{x}_2 \pm (t \cdot SE(\bar{x}_1 - \bar{x}_2))\]
\end{enumerate}

\subsection*{Analiza variance (ANOVA)}

Primerjava povprečij večih skupin (če dve skupini je ANOVA enaka T testu).

$F$ porazdelitev ima dve prostorski skopnji

Slika variance between and within.

Primer: Tri skupine desetih slučajno izbranih študentov so postavljene v tri različne učilnice. A ima konstantno glasbo v ozadju, B variabilno glasbo v ozadju, C brez glasbe. Po enem mesecu nas zanima, ali glasba pomaga pri učenju.

Izračun:
\begin{enumerate}
    \item Postavimo ničelno in alternativno hipotezo:
        \begin{itemize}
            \item $H_0$: Med skupinami ni razlik v vsrkavanju informacij.
            \item $H_1$: Med skupinami so razlike v vsrkavanju informacij.
        \end{itemize}
    \item Izračunamo povprečja. Skupno povprečje je $\bar{x} = 5,1$, povprečja posameznih skupin pa so $\bar{x}_1 = 7$, $\bar{x}_2 = 4$, in $\bar{x}_3 = 4,3$.
    \item Vsota kvadratov:
        \begin{itemize}
            \item Med skupinami: $SS_{between} = 54,6$
            \item Znotraj skupin: $SS_{within} = 90,1$
        \end{itemize}
    \item Prostostne stopnje:
        \begin{itemize}
            \item Med skupinami: $df_{between} = 2$
            \item Znotraj skupin: $df_{within} = 27$
        \end{itemize}
    \item Povprečni kvadrat:
        \begin{itemize}
            \item Med skupinami: $MS_{between} = \frac{SS_{between}}{df_{between}} = \frac{54,6}{2} = 27,3$
            \item Znotraj skupin: $MS_{within} = \frac{SS_{within}}{df_{within}} = \frac{90,1}{27} = 3,34$
        \end{itemize}
    \item Empirična F-vrednost: 
        \[F = \frac{MS_{between}}{MS_{within}} = \frac{27,3}{3,34} = 8,18\]
    \item V tabeli poiščemo kritično vrednost pri $\alpha = 5\%$: $F_{critical} = 2,03$.
    \item Ker je empirična F-vrednost (8,18) večja od kritične vrednosti (2,03), zavrnemo ničelno hipotezo $H_0$.
    \item Izračunamo eta kvadrat ($\eta^2$), ki je merilo učinka:
        \[\eta^2 = \frac{SS_{between}}{SS_{between} + SS_{within}} = \frac{54,6}{54,6 + 90,1} = 0,38\]
        \[ \eta = \sqrt{\eta^2} = \sqrt{0,38} \approx 0,62\]
\end{enumerate}

\subsection*{Neparametrične alternative}






\begin{Vaje}{1}
    ...
\end{Vaje}