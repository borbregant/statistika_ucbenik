\section{Multivariantna statistika}

Povezanost med spremenljivkama ne pomeni nujno, da med njima obstaja vzročna povezava. Spremenljivke so lahko povezane tudi navidezno in jih pojasni uvedba tretje spremenljivke (npr. poletni čas pojasni povezavo med napadi morskih psov in prodajo sladoleda).

\subsection*{Multipla regresijska analiza (linearna)}

Metoda za preučevanje razmerja med odvisno spremenljivko in dvema ali več neodvisnimi spremenljivkami (vrednosti odvisne spremenljivke napovedujemo z nizom neodvisnih spremenljivk).

Regresijski model $Y=\beta_0 +\beta_1 X_1 + \ldots + \beta_k X_k$.

Delež variabilnosti napovemo z $R^2$.

Velikost vzorca želimo vsaj 10-krat več primerov kot neodvisnih spremenljivk.

Predpostavke multiple linearne regresije:
\begin{itemize}
    \item Neodvisnost
    \item Normalnost
    \item Homoskedastičnost
    \item Linearnost
\end{itemize}

\subsection*{Multipla regresijska analiza (logistična)}

Kjer je odvisna spremenljivka nominalna.

Namesto linearne funkcije uporabimo logistične funkcije, ki zagotovijo določitev kategorije.

\subsection*{Razvrščanje v skupine (clustering)}

...

\subsection*{Metode zmanjšanja dimenzionalnosti podatkov}

PCA

Faktorksa analiza

\subsection*{Analiza zanesljivosti}

Cronbachov $\alpha$ koeficient

\subsection*{Druge multivariantne metode}

Kanonična korelacijska analiza, diskriminantna analiza, strukturni modeli...




\begin{Vaje}{1}
    ...
\end{Vaje}